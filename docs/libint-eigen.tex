\documentclass[12pt]{article}

\usepackage[onlytext]{MinionPro}
\usepackage{microtype}

\setlength{\parindent}{0in}                                                     % No indentation for every paragraph

\usepackage[left=3cm, right=3cm]{geometry}		                                % Margins left and right
\usepackage{hyperref}                                                           % Clickable table of contents in PDFs
\usepackage{datetime}                                                           % Currenttime

\usepackage{physics}
\usepackage{listings}                                                           % Code inclusion
\usepackage{color}

\definecolor{codegreen}{rgb}{0,0.6,0}
\definecolor{codegray}{rgb}{0.5,0.5,0.5}
\definecolor{codepurple}{rgb}{0.58,0,0.82}
\definecolor{backcolour}{rgb}{0.95,0.95,0.92}

\lstdefinestyle{mystyle}{
    backgroundcolor=\color{backcolour},
    commentstyle=\color{codegreen},
    keywordstyle=\color{magenta},
    numberstyle=\tiny\color{codegray},
    stringstyle=\color{codepurple},
    basicstyle=\footnotesize,
    breakatwhitespace=false,
    breaklines=true,
    captionpos=b,
    keepspaces=true,
    numbers=left,
    numbersep=5pt,
    showspaces=false,
    showstringspaces=false,
    showtabs=false,
    tabsize=2
}

\lstset{style=mystyle}

\lstset{ %
    language=C++,
    style=mystyle
}

\title{The libint-eigen interface}
\author{Laurent Lemmens}
\date{\today \hspace{6pt} \currenttime}

\begin{document}

\maketitle

\begin{center}
\line(1,0){250}
\end{center}

\tableofcontents
\newpage



\section{Basis sets}
    In the libint basis set context, an \textit{orbital} refers to a \textit{GTO} (Gaussian-type (atomic) orbital), which is just a \textit{basis function}. Orbitals/GTOs/basis functions are used to form \textit{molecular orbitals}, in which Slater determinants are expanded. Finally, the final CI wave function is written as a linear combination of Slater determinants. \\

    A (Cartesian) GTO, with angular momenta $l_x$, $l_y$ and $l_z$ and centered on $\vb{R}=(X, Y, Z)$, has the following mathematical form:
    \begin{equation}
        \phi_{\zeta, l_x, l_y, l_z, \vb{R}}(x, y, z) = N (x - X)^{l_z} (y - Y)^{l_y} (z - Z)^{l_z} \exp(-\zeta r^2) \thinspace ,
    \end{equation}
    in which
    \begin{equation}
        r^2 = x^2 + y^2 + z^2 \thinspace .
    \end{equation}

    The sum of the angular momenta
    \begin{equation}
        l = l_x + l_y + l_z
    \end{equation}
    determines the type of AO: $l=0$ refers to an s-type orbital, $l=1$ to a p-type orbital, etc. (analogous to naming of the eigenfunctions of the hydrogen atom).



\section{Are the contracted basis functions normalized?}

    When constructing a \lstinline{libint2::BasisSet} object as in say

\begin{lstlisting}
libint2::BasisSet obs ("STO-3G", atoms);
\end{lstlisting}

    the corresponding file \lstinline{sto-3g.g94} is read (which is located at \lstinline{\$LIBINT_DATA_PATH}), in which LibInt2 finds the exponents and contraction coefficients for the given basis for a given element. \\

    In \lstinline{libint2/basis.h}, we can see the following (edited for brevity) code:

\begin{lstlisting}
static ... read_g94_basis_library(...){
    ...
    ref_shells[Z].push_back(
        libint2::Shell{...}
        )
    ...
}
\end{lstlisting}

    which calls a specific constructor of \lstinline{libint2::Shell}:

\begin{lstlisting}
Shell(...) {
    // embed normalization factors into contraction coefficients
    renorm();
\end{lstlisting}

    that makes sure that the CGTO is normalized by including the normalization factor in the contraction coefficients. So, \textbf{yes}, LibInt2 internally works with normalized basis functions.



\end{document}
